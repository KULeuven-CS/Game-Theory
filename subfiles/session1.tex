\documentclass[../main.tex]{subfiles}
\begin{document}

\begin{question}
	Exercise 1-1: Provide an example of a two-player NFG in which a player has two \textbf{very} weakly dominant pure strategies, but his opponent prefers that he play one of them rather than the other.
	\textit{[\textbf{update:} After reporting the erratum, described below, the teaching assistant confirmed that the `very' was indeed missing.]}
\end{question}

\begin{solution}
	\label{twoweak}
	I think this exercise asks something impossible, probably due to the subtle difference between `\emph{very} weak dominance' and `weak dominance'.
	Two weakly dominant strategies can NEVER weakly dominate each other by definition and therefore can NOT coexist.
	We will prove this by contradiction.

	Let's start with the assumption that $A,B$ are both pure strategies of player one.
	For a strategy to be weakly dominant it will weakly dominate ANY other strategy for that agent.
	This means that we can assume that $A$ weakly dominates $B$ and vice versa.
	The set of strategy profiles of remaining players is $S_2$ and $u_1(S1,S2)$ is the utility function of player one.

	If $A$ weakly dominates $B$ for player 1 the following holds by definition \egtcite{20}:
	\begin{equation}
		\label{eq:Adom1}
		\forall s_{\--i} \in S_2 : u_1(A,s_{\--i}) \geq u_1(B,s_{\--i})
	\end{equation}
	and
	\begin{equation}
		\label{eq:Adom2}
		\exists s_{\--i} \in S_2 : u_1(A,s_{\--i}) > u_1(B,s_{\--i})
	\end{equation}

	If $B$ weakly dominates $A$ for player 1 the following ALSO holds by definition \egtcite{20}:
	\begin{equation}
		\label{eq:Bdom1}
		\forall s_{\--i} \in S_2 : u_1(B,s_{\--i}) \geq u_1(A,s_{\--i})
	\end{equation}
	and
	\begin{equation}
		\label{eq:Bdom2}
		\exists s_{\--i} \in S_2 : u_1(B,s_{\--i}) > u_1(A,s_{\--i})
	\end{equation}

	We can easily combine \autoref{eq:Adom1} and \autoref{eq:Bdom1} because they iterate over the same set:
	\begin{equation}
		\label{eq:equal}
		\forall s_{\--i} \in S_2 : u_1(A,s_{\--i}) \geq u_1(B,s_{\--i}) \geq u_1(A,s_{\--i}) \\ \Rightarrow \forall s_{\--i} \in S_2 :  u_1(A,s_{\--i}) = u_1(B,s_{\--i})
	\end{equation}

	Now \autoref{eq:equal} is in contradiction with the existence of a strategy that is \emph{strictly} better for some strategy of an opponent as required by either \autoref{eq:Adom2} or \autoref{eq:Bdom2}.
	By contradiction we find that our initial assumption of two weakly dominant strategies can never hold. $\blacksquare$

	If we assume that they meant ``two VERY weakly dominant pure strategies'' the question does have solution(s). Because equality is sufficient for this property these strategies CAN coexist.
	Under that, altered, assumption there are plenty solutions e.g.: 
        \begin{center}
    	    \begin{tabular}{|l|c|}
    	    \hline
    	    & $A_2$ \\
    	    \hline
			$A_1$ & $(0,1)$\\
    	    \hline
    	    $B_1$ & $(0,2)$\\
    	    \hline
    	    \end{tabular}
        \end{center}

\end{solution}

\begin{question}
Exercise 1-2: Provide an example of a two-player NFG in which a player has a dominant strategy that is not pure.
\end{question}

\begin{solution}
The Matching Pennies game is a good example of such a two-player NFG.
        \begin{center}
    	    \begin{tabular}{|l|c|c|}
    	    \hline
    	    & $Heads$ & $Tails$ \\
    	    \hline
    	    $Heads$ & $(1,-1)$ & $(-1,1)$ \\
    	    \hline
    	    $Tails$ & $(-1,1)$ & $(1,-1)$ \\
    	    \hline
    	    \end{tabular}
        \end{center}
This game obviously has no pure dominant strategy.
We know that this game has a Nash Equilibrium in which both players choose Heads and Tails with 50\% chance as strategy.
%TODO:Fix reasoning. Show that any mixed strategy dominates a pure strategy
The corresponding strategy profile for this equilibrium is $\big((0.5,0.5), (0.5,0.5)\big)$.
% The expected payoff for player one playing the pure strategy Heads is $0.5*1 + 0.5* -1 = 0$.
% The expected payoff for player one playing the pure strategy Tails is $0.5*-1 + 0.5* 1 = 0$.
% The expected payoff for player one when playing the mixed $(0.5,0.5)$ strategy is $0.5*(0.5*1 + 0.5* -1) + 0.5*(0.5*-1 + 0.5* 1)= 0$.
Because of this, the Matching Pennies game satisfies the question (not one, but even both players have a mixed dominant strategy).
\end{solution}

\begin{question}
Exercise 1-3 Cournot duopoly\\
In a Cournot duopoly, two firms produce identical products. They each have two discrete production levels, high or low. If they produce at high production, they have a lot of the good to sell, while at low production they have less to sell. The market price of the product depends on the amount of available product, which depends on the production level of both firms. In the case where both firms cooperate 1 and produce at the low production level, the product is rare and fetches a high price on the market. In contrast, when they both produce at the high production level, the price of the product drops significantly and both firms have lower profits than when cooperating. In the case where only one of the firms produces at the high production level, the decrease in the price of the product is compensated by higher sales, increasing the profits of the firm that produces at a high level such that its profit are higher than when cooperating. The competitor is then even worse off than when also producing at high production.\\

Answer the following questions.
\begin{enumerate}
	\item Analyse the preferences of the players.
	\begin{enumerate}
		\item List all possible outcomes of the interaction described above.
		\item Describe each player’s preferences over the set of outcomes.
	\end{enumerate}
	\item Model this interaction as a game.
	\begin{enumerate}
		\item List the possible actions of each player in this game.
		\item Describe each player’s preferences over the set of outcomes.
		\item Write down the game in matrix form. Suppose that a company’s utility equals its utility and that producing alone yields \euro 100 profit for one firm and \euro10 for the other, while both producing at low / high production yields \euro70 / \euro30 for both firms.
	\end{enumerate}
	
	\item Nash equilibria
	\begin{enumerate}
		\item List all Nash equilibria (i.e. list the Nash equilibria and argue why there are no others). 
		\item Could you have derived the Nash equilibria from the preference relation alone (i.e. does the additional information in 2b. influence your answer this question)?
	\end{enumerate}
	
	\item Pareto-optimality
	\begin{enumerate}
		\item Determine all Pareto-optimal outcomes.
		\item Are there Pareto-optimal outcomes that you prefer over others? Why?
	\end{enumerate}		
		
	\item Does this game remind you of another game?
\end{enumerate}
\end{question}

\begin{solution}
\begin{enumerate}
	\item Analyse the preferences of the players.
	\begin{enumerate}
		\item O1: (low, low)\\
				O2: (high, low)\\
				O3: (low, high)\\
				O4: (high, high)\\
		\item  P1: $O2 > O1 > O4 > O3$ \\
			P2: $O3 > O1 > O4 > O2$
	\end{enumerate}
	\item Model this interaction as a game.
	\begin{enumerate}
		\item A: high, B: low 
		\item	\begin{tabular}{|l|c|r|}
		    	    \hline
		    	    & $A_2$ & $B_2$ \\
		    	    \hline
		    	    $A_1$ & $(30,30)$ & $(100,10)$ \\
		    	    \hline
		    	    $B_1$ & $(10,100)$ & $(70,70)$ \\
		    	    \hline
		    	\end{tabular}
		       
	\end{enumerate}
	
	\item (30,30)
	
	\item 
	\begin{enumerate}
		\item (70,70), (100,10) and (10,100)
		\item yes
	\end{enumerate}		
		
	\item Prisoner's dilemma
\end{enumerate}
\end{solution}

\begin{question}
Exercise 1-4 To pay or not to pay\\

Every day, railway customers face the decision of whether or not to buy a train ticket. Likewise, the railway company has to decide on whether or not to check if customers are in possession of a ticket. We assume that the objective of both the customer and the railway company is to optimize their own costs/profits. In the case where customers are in the possession of a ticket and the railway company checks, the railway company regrets checking (lost effort). In the case where customers are in the possession of a ticket and the railway company does not check, the customer regrets buying a ticket (buying a ticket was unnecessary). In the other cases, either the railway company or the customer is happy and the other party is unhappy. The railway company prefers being able to fine a customer over a paying customer that they do not have to check.\\

Answer the following questions.
\begin{enumerate}
	\item Analyse the preferences of the players.
	\begin{enumerate}
		\item List all possible outcomes of the interaction described above.
		\item Describe each player’s preferences over the set of outcomes.
	\end{enumerate}
	\item Model this interaction as a game.
	\begin{enumerate}
		\item List the possible actions of each player in this game.
		\item Describe each player’s preferences over the set of outcomes.
		\item Write down the game in matrix form. Assume that checking costs the railway company \euro1, that a ticket costs \euro10 and that the fine is \euro20.
	\end{enumerate}
	
	\item Nash equilibria
	\begin{enumerate}
		\item List all Nash equilibria (i.e. list the Nash equilibria and argue why there are no others). 
		\item Could you have derived the Nash equilibria from the preference relation alone?
	\end{enumerate}
	
	\item Determine all Pareto-optimal outcomes.
		
\end{enumerate}


\end{question}

\begin{solution}
\begin{enumerate}
	\item Analyse the preferences of the players.
	\begin{enumerate}
		\item List all possible outcomes of the interaction described above.
		\item Describe each player’s preferences over the set of outcomes.
	\end{enumerate}
	\item Model this interaction as a game.
	\begin{enumerate}
		\item List the possible actions of each player in this game.
		\item Describe each player’s preferences over the set of outcomes.
		\item The game bellow assumes that the traveler regrets paying if there is no check with a cost of $2$ (resulting in $-12$ cost for the traveler in that case):
			\begin{center}
				\begin{tabular}{|r|c|c|}
					\hline
					& $C(heck)$ & $NC(heck)$ \\
					\hline
					$P(ay)$ & $(-10,9)$ & $(-12,10)$ \\
					\hline
					$NP(ay)$ & $(-20,19)$ & $(0,0)$ \\
					\hline
				\end{tabular}
			\end{center}
	\end{enumerate}
	
	\item Nash equilibria
	\begin{enumerate}
		\item To find all the Nash Equilibria we first look at the pure strategies.
			It's clear that there are no pure strategy Nash equilibria. If one should exists there should be a cell for which both players consider that strategy their best response. If we markt the pure strategy best responses ($\underline{underline}$ for traveler, $\overline{overline}$ for railway) we can easiliy show that this is not the case:\\
		 \setlength\extrarowheight{1pt}
		 \begin{center}
			 \begin{tabular}{|r|c|c|}
				 \hline
				 & $C(heck)$ & $NC(heck)$ \\
				 \hline
				 $P(ay)$ & $(\underline{-10},9)$ & $(-12,\overline{10})$ \\ 
				 \hline
				 $NP(ay)$ & $(-20,\overline{19})$ & $(\underline{0},0)$ \\ 
				 \hline
			 \end{tabular}
		 \end{center}

		 Let's now look for the mixed strategy Nash equilibibria.
		 This means each player will play one of their pure strategies with a certain probablitity.
		 The probablity used by one player must make all other players indifferent between their pure strategies otherwise they would never use a mixed strategy \egtcite{12}. For an extensive explaination see \cite{mixedstr}. Lets introduce the probabilities $q$ for the rowplayer and $p$ for the column player:

			\begin{center}
				\begin{tabular}{|r|c|c|}
					\hline
					& $p : C(heck)$ & $1-p :NC(heck)$ \\
					\hline
					$q:P(ay)$ & $(-10,9)$ & $(-12,10)$ \\
					\hline
					$1-q: NP(ay)$ & $(-20,19)$ & $(0,0)$ \\
					\hline
				\end{tabular}
			\end{center}

		To determine rowplayer strategy ($q$) we now know that the column player should be indifferent between player $C$ and $NC$ or in symbols:
		$$U_C = U_{NC}$$
		The expected utilities are dependant on the payoffs of the column player and the probability $q$:
		$$U_C = 9q + 19 (1-q)$$
		$$U_{NC} = 10 q + 0 (1-q)$$
		We can now easily solve for $q$:
		$$U_C = 9q + 19 (1-q) = 10 q + 0 (1-q) = U_{NC}$$
		$$ 9q + 19- 19q = 10 q$$
		$$ 19 = 10 q - 9q + 19q$$
		$$ 19 = 20 q$$
		$$ \frac{19}{20} = q$$


		We can determine the strategy of the rowplayer analogously (solving for $p$):
		$$U_P = -10p - 12 (1-p) = -20 p + 0 (1-p) = U_{NP}$$
		$$ -10p - 12 +12p = -20 p $$
		$$ -12 = -20 p +10p -12p$$
		$$ -12 = -22 p$$
		$$ \frac{-12}{-22} = \frac{6}{11} = p$$

		The result gives us the weak Nash equilibrium: $$s=\Big((\frac{19}{20},\frac{1}{20}),(\frac{6}{11},\frac{5}{11})\Big)$$
		Because this equilibrium contains the best responses of all players in all (rational) situations this equilibrium is unique. So the game has only a single Nash Equilibrium. \\(\textit{Is there a better way to verify/prove this?})


	\item It would be impossible to derive the Nash Equilibria from the preference relation alone because we clearly used the payoff values to determine the Nash Equilibirum. It would, however, be possible to determine (the non-existance of) \emph{pure} Nash equilibria from the preference relation alone.
	\end{enumerate}
	
	\item Determine all Pareto-optimal outcomes.
	All four possible strategies are Pareto-optimal. It is easy to see that for each possible strategy, no other strategy exists that generates a higher payoff for one player while not generating a lower payoff for the other player.
		
\end{enumerate}

\end{solution}

\begin{question}
Exercise 1-5 Making a calculated guess\\
Consider the $n$-player game where each player has to name a natural number $x$ with $ 0 \leq x \leq 100$ and the player that is closest to two thirds of the average guess becomes the winner (say he gets a prize); all of the other players are losers (they get nothing). In the case of a tie, the prize is split among the winners. Consequently, players strictly prefer to win alone.\\

\begin{enumerate}
\item Analyse the preferences of the players.
	\begin{enumerate}
	\item Is it feasible to write down all outcomes of the interaction like you did before?
	\item Describe each player’s preferences over the set of outcomes.
	\end{enumerate}
\item Model this interaction as a game.
	\begin{enumerate}
	\item For each player, what are the available actions in this game?
	\item Can you write down the game in matrix form? If feasible, do so.
	\end{enumerate}
\item Nash equilibria
	\begin{enumerate}
	\item For $n = 2$, list all Nash equilibria; prove that there are no others.
	\item For general $n$ list all Nash equilibria; prove that there are no others.
	\end{enumerate}	
\item Answer the questions in part III again for the following variations of the game.
	\begin{enumerate}
	\item In case of a tie, a single winner is picked randomly from the set of tied players.
	\item In case of a tie, there are no winners (consider only the case $n=2$).
	\end{enumerate}	
\end{enumerate}
Note: Even though the result of 3b might seem obvious, its proof is nontrivial.
\end{question}

\begin{solution}

\end{solution}

\begin{question}
Let $\Gamma_N$ and $\Gamma'_N$ be NFGs and let $\Gamma_N$  be identical to $\Gamma'_N$ except for the utility fuction of player $i$, which is the same in both games up to a positive affine transformation.
Denote the set of best response strategies of $i$ against strategy profile $s_{\--i}$ by the other players in $\Gamma_N$  as $BR_{\Gamma_N}(s_{\--i})$.
Argue that $BR_{\Gamma_N}(s_{\--i}) = BR_{\Gamma'_N}(s_{\--i})$ for all possible $s_{\--i}$.
What does this tell us about strict/weak domination relation between strategies and the set of Nash equilibria in $\Gamma_N$ and $\Gamma'_N$? Are $\Gamma_N$ and $\Gamma'_N$ equal?

The set of positive affine transformations of utility function $u_i(s)$ consists of all utility functions of the form $au_i(s) + b$ where $a \in \mathbb{R}, a > 0$ and $b \in \mathbb{R}$.
\end{question}

\begin{solution}

\end{solution}

\begin{question}
	Prove that if a player has a strictly dominant strategy $\overline{s}_i$, then that player plays $\overline{s}_i$, in all Nash equilibria. 
\end{question}

\begin{solution}
	We will prove this statement by contradiction.
	Let's assume that $\overline{s}_i$ is a strictly dominant strategy for player $i$ ($\overline{s}_i \in S_i$) that is not played in some Nash equilibrium $N = (s_1,\ldots,s'_i,\ldots,s_n)$ because $s'_i \neq \overline{s}_i$ is played there instead.

	From the definition of a Nash equilibrium \egtcite{11} we find that $s'_i$ is a best response to $s_{\--i}$.
	The definition of a Best response of player $i$ to the strategy profile $s_{\--i}$ states that $s'_i \in S_i$ is a mixed strategy such that:
	\begin{equation}
		\label{eq:bestr}
		\forall s_i \in S_i : u_i(s'_i,s_{\--i}) \geq u_i(s_i,s_{\--i})
	\end{equation}
	In particular the following also holds for the strictly dominant strategy since $\overline{s}_i \in S_i$:
	\begin{equation}
		\label{eq:uneq1}
		u_i(s'_i,s_{\--i}) \geq u_i(\overline{s}_i,s_{\--i})
	\end{equation}
	Because $\overline{s}_i$ is strictily dominant the following holds \egtcite{20}:
	\begin{equation}
		\forall s_i \in S_i/\{\overline{s}_i\}: \forall s_{\--i} \in S_{\--i} : u_1(\overline{s}_i,s_{\--i}) > u_1(s_i,s_{\--i})
	\end{equation}
	And in particular since $\overline{s}_i \neq s'_i \Rightarrow s'_i \in S_i/\{\overline{s}_i\}$:
	\begin{equation}
		\label{eq:uneq2}
		 u_1(\overline{s}_i,s_{\--i}) > u_1(s'_i,s_{\--i})
	\end{equation}
	If we now cominbe \autoref{eq:uneq1} and \autoref{eq:uneq2}:
	\begin{equation}
		 u_1(\overline{s}_i,s_{\--i}) > u_1(s'_i,s_{\--i}) \geq u_i(\overline{s}_i,s_{\--i})
	\end{equation}
	This unequality clearly can not hold under the given assumptions.
	Indeed if instead $s'_i = \overline{s}_i$ \autoref{eq:uneq2} can not be deducted (since $\overline{s}_i$ is exempted from $S_i$) and in that case it's trivial to see that $u_1(s'_i,s_{\--i}) \geq u_i(\overline{s}_i,s_{\--i})$ does hold.
	By condradiction we proved that if a player has a strictly dominant strategy $\overline{s}_i$ it will always be played in any Nash equilibrium. $\blacksquare$ 
\end{solution}

\begin{question}
	Prove that, if a player has two \textbf{very} weakly dominant strategies, then for every strategy choice by her opponents the two strategies yield him equal payoffs.
	\textit{[\textbf{update:} After reporting the erratum, described below, the teaching assistant confirmed that the `very' was indeed missing.]}
\end{question}

\begin{solution}
	Again I assume that it should be ``VERY weakly dominant strategies'' because a solution would be impossible otherwise (see \autoref{twoweak}).
	Let's call the very weakly dominant strategies for player 1 $A$ and $B$.
	The set of strategy profiles of remaining players is $S$ and $u_1(S1,S2)$ is the utility function of player one.
	If $A$ very weakly dominates $B$ the following holds by definition \egtcite{20}:
	\begin{equation}
		\label{eq:vw1}
		\forall s_{\--i} \in S : u_1(A,s_{\--i}) \geq u_1(B,s_{\--i})
	\end{equation}
	If $B$ very weakly dominates $A$ the following ALSO holds by definition \egtcite{20}:
	\begin{equation}
		\label{eq:vw2}
		\forall s_{\--i} \in S : u_1(B,s_{\--i}) \geq u_1(A,s_{\--i})
	\end{equation}
	Combining \autoref{eq:vw1} and \autoref{eq:vw2} we find:
	\begin{equation}
		\forall s_{\--i} \in S : u_1(A,s_{\--i}) \geq u_1(B,s_{\--i}) \geq u_1(A,s_{\--i})
	\end{equation}
	This equation can only hold if $u_1(A,s_{\--i}) = u_1(B,s_{\--i})$ for every strategy chosen by the opponents. \textit{Q.E.D.}
\end{solution}

\begin{question}
	Prove that strictly dominant strategies are always pure. 
\end{question}

\begin{solution}

\end{solution}

\begin{question}
	Guessing right

	Player 1 and 2 each choose a member of the set $\{1,\ldots,K\}$. If the players choose the same number then player 2 pays \$1 to player 1; otherwise no payement is made. Each player maximizes his expected monetary payoff. Formulate this situation as a strategic game and find its Nash equilibria.
\end{question}

\begin{solution}

\end{solution}

\end{document}
