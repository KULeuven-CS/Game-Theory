\documentclass[../main.tex]{subfiles}
\begin{document}

\begin{question}
    Exercise 2-1: Order of strategy removal.
    Consider the following game:
    
    \begin{center}
	    \begin{tabular}{|l|c|r|}
	    \hline
	    & Left & Right \\
	    \hline
	    Top & (1,1) & (0,1) \\
	    \hline
	    Middle & (1,2) & (2,1) \\
	    \hline
	    Bottom & (0,0) & (2,1) \\
	    \hline
	    \end{tabular}
    \end{center}
    
    \begin{enumerate}
    	\item Find all pure strategy Nash equilibria (PSNE) in the above normal form matrix.
    	\item List all strategies that are weakly dominated by another strategy.
    	\item In this game, can a different order of removal lead to different outcomes?
    	\item List all pure strategy profiles that are not PSNE of the full game but become PSNE after iterative removal of weakly dominated strategies.
    	\item List all pure strategy profiles that are PSNE of the full game but can never be PSNE after iterative removal of weakly dominated strategies, regardless of order.
    \end{enumerate}
   
\end{question}

\begin{solution} 

    \begin{enumerate}
    	\item 
    		\begin{itemize}
	    		\item Top-Left: (1,1)
	    		\item Middle-Left (1,2)
	    		\item Bottom-Right (2,1)
    		\end{itemize}
    	\item
    		All strategies on the Top and bottom rows are weakly dominated. (Note from Dennis:  Bottom-Right (2,1) and Top-Left (1,1) are not weakly dominated, they are only very weakly dominated.)
    	\item Yes: removing the bottom row first and then the last column (a) will result in a different outcome than removing the top row first, then the bottom row and finally the last column (b).
    	\item We consider the two different results of the iterative removal of dominated strategies separately.
			\begin{enumerate}[(a)]
				\item The first result is: \\
					\begin{center}
						\begin{tabular}{|l|c|}
							\hline
							& Left \\
							\hline
							Top & (1,1) \\
							\hline
							Middle & (1,2) \\
							\hline
						\end{tabular}
					\end{center}
					Only PSNEs that already exist in the initial game remain so for this cases there are no new PSNEs.
			\item  The second result is: \\
				\begin{center}
					\begin{tabular}{|l|c|}
						\hline
						& Left \\
						\hline
						Middle & (1,2) \\
						\hline
					\end{tabular}
				\end{center}
				Again only a full game PSNE remains so there are no additional PSNEs.
			\end{enumerate}
    	\item We can see from the previous question that only (Bottom, Right) can never be a PSNE after iterative removal of weakly dominated strategies. 
    \end{enumerate}
\end{solution}

\begin{question}
	Consider an arbitrary two-player normal form game $G = (N=\{P_1,P_2\},\{A_1,A_2\},\{u_1,u_2\})$.
	Let $i \in N$ and her maxmin value be $w_i$ and minmax value be $v_i$. Prove that $v_i \geq w_i$.
\end{question}

\begin{solution}
	We will prove $v_i \geq w_i$ using the definitions of maxmin and minmax.

	\noindent The definition of maxmin $w_i$ in symbols is \egtcite{15}:
	\begin{equation}
		w_i = \max_{s_i} \min_{s_{\--i}} u_i(s_i,s_{\--i}) = u_i(s_{i}^{*},s_{-i}^{*})
	\end{equation}
	\noindent and where \--i indicates the opposing player. 

	$$v_i = \min_{s_{\--i}} \max_{s_i} u_i(s_i,s_{\--i})$$

	\noindent Following from the definition of $\max_{s_i}$:
	$$\max_{s_i} u_i(s_i,s_{\--i}) \geq u_i(s_i^{*}, s_{\--i})$$
	\noindent In particular this also holds for:
	$$\max_{s_i} u_i(s_i,s_{\--i}) \geq u_i(s_i^{*},s_{\--i}^{*}) = w_i$$
	\noindent And if the left-hand side is already bigger than $w_i$ so will $v_i$.
\end{solution}

\begin{question}
Exercise 2-3 Beer Sales at Football Stadium.

There are five collinear seating sections located along the lower level of one end of a football stadium, designated sections 101 through 105. Two vendors, Butt and Stella, each have obtained a license to sell beer from a fixed location in a seating section of their choice. The vendors can be located in the same or different sections. There are 250 fans in each section, and each fan will purchase a single beer from the beer stand that is closest to his or her section (they will not buy beer from the other stand). If the two beer stands are
equidistant from a seating section, then half the fans in that section will purchase form one stand and half from the other. Each beer sold results in a profit of \$1.

	\begin{enumerate}
		\item Construct a normal form matrix with associated payoffs.
		\item What strategies survive iterated removal of strictly dominated strategies?
		\item What if it is not possible for either vendor to set up shop in section 103?
		\item Do the answers to (1) or (2) change if each vendor must pay a licence fee of \$250?
		\item Do the answers to (1) or (2) change if each vendor must pay a licence fee of \$250 for sections 101 and 105, \$350 for sections 102 and 104 and \$450 for section 103? If so, how?
	\end{enumerate}

\end{question}



\begin{solution}

	
	\begin{enumerate}
		\item 
			\begin{tabular}{|l|c|c|c|c|r|}
				\hline
				Butt/Stella & 101 & 102 & 103 & 104 & 105\\
				\hline
				101 & (625,625) & (250,1000) & (375,875) & (500,750) & (625,625)\\
				\hline
				102 & (1000,250) & (625, 625) & (500,750) & (625,625) & (750,500)\\
				\hline
				103 & (875,375) & (750,500) & \cellcolor{yellow!25}(625,625) & (750,500) & (875,375)\\
				\hline 
				104 & (750,500) & (625,625) & (500,750) & (625,625) & (1000,250)\\
				\hline
				105 & (625,625) & (500,750) & (375,875) & (250,1000) & (625,625)\\
				\hline
			\end{tabular}
		\item Only 103-103: (625,625)
		\item 
			\begin{tabular}{|l|c|c|c|r|}
				\hline
				Butt/Stella & 101 & 102 & 104 & 105\\
				\hline
				101 & (625,625) & (250,1000) & (500,750) & (625,625)\\
				\hline
				102 & (1000,250) & \cellcolor{yellow!25}(625, 625) & \cellcolor{yellow!25}(625,625) & (750,500)\\
				\hline 
				104 & (750,500) & \cellcolor{yellow!25}(625,625) & \cellcolor{yellow!25}(625,625) & (1000,250)\\
				\hline
				105 & (625,625) & (500,750) & (250,1000) & (625,625)\\
				\hline
			\end{tabular}
		\item No.
		\item Yes, table changes, equilibrium found in 103-103: (175,175)
			
	\end{enumerate}
\end{solution}

\begin{question}
	Minimax.\\
	Consider the following two games. Suppose that player 2 does not really care about her own payoff, but instead wants to minimize the payoff of player 1 (i.e. she plays her minimax strategy). Player 1 wants to maximize her minimum payoff (i.e. she plays her maximin strategy).
	
	\begin{center}
	\begin{tabular}{|l|c|c|r|}
	    \hline
	    & Left & Front & Right \\
	    \hline
	    Top & (8,-8) & (7,-7) & (3,-3) \\
	    \hline
	    Middle & (5,-5) & (9,-9) & (4,-4) \\
	    \hline
	    Bottom & (1,-1) & (6,-6) & (2,-2) \\
	    \hline
	\end{tabular}
	\quad
	\begin{tabular}{|l|c|c|r|}
	    \hline
	    & Left & Front & Right \\
	    \hline
	    Top & (8,-8) & (7,-7) & (3,-3) \\
	    \hline
	    Middle & \textbf{(4,-4)} & (9,-9) & \textbf{(5,-5)} \\
	    \hline
	    Bottom & (1,-1) & (6,-6) & (2,-2) \\
	    \hline
	\end{tabular}
	\end{center}
	
	\begin{enumerate}
		\item Which of the above matrices has a solution in pure strategies? What is the safety level of player 1 in this game?
		\item The other matrix has a solution in mixed strategies. What is it? What is the safety level of player 1 in this game?
		\item Both of these 3 $\times$ 3 matrices map different action profiles to different payoffs in the set $\{1, ... ,9\}$.
		Let G be the set of all such games. What is the largest maximin value achieved by player 1 for any
		game in G? What is the smallest? Give two matrices for which these maximin values are achieved.
	\end{enumerate}
	
\end{question}

\begin{solution}
In a zero-sum game, the minimax solution is the same as the Nash Equilibrium
\begin{enumerate}
	\item 
	$	\begin{bmatrix}
	  		8 & 7 & 3 \\
	  		5 & 9 & \textbf{4} \\
	  		1 & 6 & 2
	 	\end{bmatrix}
		\quad
		\begin{bmatrix}
			\textbf{8} & 7 & \textbf{3} \\
			\textbf{4} & 9 & \textbf{5} \\
			1 & 6 & 2
		 \end{bmatrix}
	$	 
	\item After iterative elimination of dominated strategies: 
	$
		\begin{bmatrix}
		(8,-8) & (3,-3) \\
		(4,-4) & (5,-5)
		\end{bmatrix} \\
	$	
	$ 8q + 3(1-q) = 4q + 5(1-q)
		 \Leftrightarrow q = 1/3, p = 1/6$\\
	The solution is $(1/6,1/3)$.
	
	\item
	Smallest: 
	$	\begin{bmatrix}
		 	1 & 4 & 7 \\
		 	2 & 5 & 8 \\
		 	\textbf{3} & 6 & 9
		 \end{bmatrix}$\\
	Largest:
	$	\begin{bmatrix}
		 	1 & 2 & 3 \\
		 	4 & 5 & 6 \\
		 	\textbf{7} & 8 & 9
		\end{bmatrix}$\\
\end{enumerate}
\end{solution}



\begin{question}
	In the Cournot duopoly model of last week, companies only produced at two discrete production levels: high and low. We now model the interaction when companies produce at arbitrary production levels.
	
	Model the interaction as a strategic game $(\{1,2\},\{A_1,A_2\},\{u_1,u_2\})$ in which $A_i = [0, +\infty)$ and $a_i \in A_i$ represents how many units of product company $i$ produces. Let the price that a company receives for each unit be equal to $max\{C - bq, 0\}$, where $q$ is the total units of product on the market and $b > 0$. Further assume that producing a unit of product costs $c_1$ for company one and $c_2$ for company two ($c_1 > 0, c_2 > 0)$.
	
	\begin{enumerate}
	\item Write down the utility function of each company.
	\item Derive a formula for the PSNE analytically.
	\item Determine the equilibrium for $C = 1$, $b = 1$, $c_1 = c_2 = 0$ and for $C = 10$, $b = c_1 = 1$, $c_2 = 2$.
	\item Illustrate one of the equilibria above graphically.
	\item Argue why there are no mixed Nash equilibria. \
	[Hint: Take cases on the strategy played by your opponent: pure or mixed. Show that in both cases, the best response is a pure strategy].
	\end{enumerate}
\end{question}



\begin{solution}
	\begin{enumerate}
	\item
		\begin{align}
		u_1(a_1,a_2)& = a_1(max\{0,C-b(a_1+a_2)\} - c_1)\\
		u_2(a_1,a_2)& = a_2(max\{0,C-b(a_1+a_2)\} - c_2)
		\end{align}
		
		
	\item 
		\begin{align}
		u_1& = a_1(C-b(a_1+a_2) - C_1)\\
		u_1^{'}& = C - 2ba_1 - ba_2 - c_1\\
		\end{align}
		
		\begin{align}
		C - 2ba_1 - ba_2 - c_1& = 0\\
		\Leftrightarrow C-ba_2-c_1& = 2ba_1
		\end{align}
		
		\begin{align}
		a_1& = \frac{C-ba_2 - c_1}{2b}\\
		a_2& = \frac{C - ba_1 - c_2}{2b}\\
		a_1& = \frac{C+C_2 - 2C_1}{3b}
		\end{align}
		
	\item
		$(\frac{1}{3},\frac{1}{3})$ for $C = 1$, $b = 1$, $c_1 = c_2 = 0$\\
		$(\frac{10}{3},\frac{7}{3})$ for $C = 10$, $b = c_1 = 1$, $c_2 = 2$
	
	\item grafiek %TODO grafiek
	\item ontbreekt %TODO
	\end{enumerate}
\end{solution}

\begin{question}
Exercise 2-6 Iterative elimination of weakly dominated strategies and PSNE
\end{question}

\begin{solution}
\end{solution}

\begin{question}
Exercise 2-7 Order of iterative elimination of strictly dominated strategies.
\end{question}

\begin{solution}
\end{solution}

\begin{question}
Exercise 2-8
\end{question}

\begin{solution}
\end{solution}

\begin{question}
Exercise 2-9
\end{question}

\begin{solution}
\end{solution}


\end{document}
